\documentclass[binding=0.6cm, oneside]{sapthesis}%remove "english" for a thesis written in Italian
%Bachelor's (laurea triennale) thesis : Lau
\usepackage[english]{babel} %use this package for a thesis written in Italian
\usepackage[utf8]{inputenx}
\usepackage{indentfirst}
\usepackage{microtype}
\usepackage{amsmath}
\usepackage{amssymb}
\usepackage{float}
\usepackage{titlesec}
\usepackage{lettrine}
\usepackage{caption}
\usepackage{subcaption}
\usepackage{listings}
\usepackage{adjustbox}
\usepackage[toc,page]{appendix}
\usepackage{listliketab}
\usepackage{csquotes}
\usepackage{tabularx}% Loads the array package
%\linespread{0.9}
\usepackage[nottoc, notlof, notlot]{tocbibind}
%\onehalfspacing
%\counterwithout{footnote}{chapter}
\usepackage{hyperref}
\hypersetup{		
			colorlinks=true,
			linkcolor=red,
                        linktoc=page,
			anchorcolor=black,
			citecolor=red,
			urlcolor=blue,
			pdftitle={The Bugs We Left Behind: Evaluating Autonomous Fuzzing Infrastructures for Overlooked Bugs and Vulnerabilities},
			pdfauthor={Matteo Sabatini},
			pdfkeywords={thesis, sapienza, roma, university}
}

%Defining macros for comments
\usepackage{xcolor}
\usepackage{xspace}
\newcommand{\mytext}[2]{\textcolor{#1}{#2}}

\newcommand{\tofix}[1]{\mytext{red}{#1}}

\newcommand{\mynote}[2]{\xspace\fbox{\bfseries\sffamily\scriptsize{#1}}
	{\small$\blacktriangleright$\textsf{\scriptsize\emph{#2}}$\blacktriangleleft$}}
\newcommand{\matteo}[1]{\mytext{magenta}{\mynote{MM}{#1}}}
\newcommand{\ziosaba}[1]{\mytext{cyan}{\mynote{ZS}{#1}}}


\title{The Bugs We Left Behind: Evaluating Autonomous Fuzzing Infrastructures for Overlooked Bugs and Vulnerabilities}
\author{Matteo Sabatini}
\IDnumber{1794627}
\course{Cybersecurity}
\courseorganizer{Facolt\`a di Ingegneria dell'informazione, informatica e statistica}
\submitdate{2023/2024}
\copyyear{2024}
\advisor{Prof. Daniele Cono D'Elia}
\coadvisor{Dott. Matteo Marini}
\authoremail{sabatini.1794627@studenti.uniroma1.it}
%Commenting this will print "Thesis not yet defended", which is good for submitting it. Then you may print it once you know the date and who the professors are
%\examdate{25/01/2025}
%\examiner{Prof. Whatever}
%\examiner{Prof. Daniele Cono D'Elia}
%\examiner{Prof. ??}

%we refer to http://ctan.mirrorcatalogs.com/macros/latex/contrib/sapthesis/sapthesis-doc.pdf for an exhaustive description of the sapthesis documentclass.




\begin{document}


%\setlength{\parindent}{0pt}    %elimina l'indentazione leggermente a destra per ogni nuovo capitolo/paragrafo
\frontmatter
\maketitle

\begin{acknowledgments}
I would like to express my gratitude to my thesis advisor, Prof. Daniele Cono D'Elia, for giving me the opportunity to deepen my knowledge in one of my favorite and also most important aspects of cybersecurity. His guidance, support and encouragement have been fundamental in the creation of this work.
\newline
I also profoundly thank my thesis co-advisor, Dr. Matteo Marini, for his expertise, suggestions and patience throughout the development of this thesis.
\end{acknowledgments}


\begin{abstract}

	\matteo{First draft}

	In a world that relies on digital systems, the security of software applications is a crucial property to protect both people and their assets. To this end, many techniques have been proposed that can detect a wide range of vulnerabilities before they can be exploited by malicious entities. The most prominent example of such techniques is \textit{Fuzz Testing}; it works by executing the software many times, generating new inputs each time to exercise as many functionalities of the code as possible. The key idea behind it is that, by producing many inputs, the \textit{fuzzer} will eventually produce one that is able to trigger a bug.

	Thanks to their \textit{fuzzing strategies}, fuzzers shine in producing inputs that allow them to execute new code as testing time passes. While reaching constantly increasing \textit{coverage}, fuzzers may miss bugs that do not cause observable behaviors at a high level. To tackle this issue, fuzzing is often used in combination with software sanitizations, a technique that detects specific silent bugs at run-time. The combination of the two techniques has become the most popular approach to detecting bugs in software.
   
	To ease the deployment of fuzzing, fuzzing frameworks are being proposed. These frameworks are fully autonomous and allow for \textit{continuous testing}. While they usually rely on a private disclosure mechanism for new bugs, these frameworks publicly release lots of data that can be analyzed. With all this accessible data, no bug should be overlooked, or an attacker could leverage the available data to detect it with a much lower effort than running fuzzing campaigns, and potentially exploit it to attack the target software.
	   
	In this thesis, we are going to evaluate fuzzing frameworks for possibly overlooked bugs. We will be exploring two main directions: first, we are going to evaluate fully autonomous mechanisms, where suboptimal configurations or procedures can cause some bugs to be missed; then, we are going to evaluate another class of frameworks where the human factor is crucial in the detection of new bugs and vulnerabilities.
	We will show that, unfortunately, these frameworks are indeed missing a relevant amount of bugs. These bugs can be detected starting from publicly available data with low effort and many of them have security implications. After reporting all the bugs we found during our evaluation, we contacted the maintainers of such frameworks and proposed new approaches to limit the amount of overlooked bugs.


\ziosaba{Q1: Which are the two frameworks mentioned? I'm not sure that I'm following which are the two directions, assuming you're referring to OSS-Fuzz and FuzzBench}

\ziosaba{Q2: "We contacted the maintainers of such framework" is something that I have yet to do, or am I missing something? We contacted the developers of the project, not the maintainers at Google}

\end{abstract}




\tableofcontents


\titleformat{\chapter}[display]  
{\normalfont\Huge\bfseries}{\chaptertitlename\ \thechapter}{20pt}{\huge}  
\titlespacing{\chapter}{0pt}{0pt}{0pt}  

\mainmatter




\chapter{Introduction}

\matteo{Brief introduction, maybe talk about software, software security and why it is important}

\section{Context}

\matteo{Talk about software testing (look also for something that is not only fuzzing, like symbolic execution or other testing technique) and then fuzz testing}

\section{Thesis Idea and Contributions}

\matteo{Briefly explain what this thesis will talk about, how we will do that and give a hint of the final results (e.g. \textit{we will show that we can find many bugs with our methodology}). Then list the contributions}

\matteo{We will come back to contributions later, just draft a list and I will attempt to fix it later.}


\section{Outline}

\matteo{Briefly describe how this thesis is structured}

\chapter{Background}

\matteo{Use this space to introduce the chapter. "In this chapter we will provide the necessary background for this thesis. We will start by introducing how xxx works, with an in-depth analysis of its yyy component and its effects on zzz. Then, we will discuss www [...]. Finally, we will conclude by detailing the inner working of vvv [...]}

\matteo{Push all this text in the Fuzz Testing section. You are introducing fuzz testing here and then you start the fuzz testing section with much information already given.}

\matteo{Once you push this text below, be careful not to repeat the same concept too many times.}

In the context of software development, \textit{fuzz testing} (or \textit{fuzzing}) refers to an automated software technique which focuses on providing random data to a program, with the objective of creating invalid or unexpected inputs that may potentially trigger crashes, assertions or memory leaks.
\newline 
This also allows developers to test their programs for so called "corner-cases", meaning situations that are hard or complex to reproduce as they should not occur when the program is being properly used, but still that could lead to unexpected and/or potential malicious behavior and thus should be properly handled. 
\newline \newline
Most fuzzers take structured inputs as reference, that will be used to generate inputs that are "valid enough" to be accepted by the program, but the strategies applied to generate such new inputs can heavily influence the effectiveness of the tests as well as the code-coverage achieved.
\newline
For this reason, fuzzers can be categorized using the 3 following characteristics:
\begin{itemize}
    \item how new inputs are generated (\textit{generation-based} or \textit{mutation-based})
    \item whether they are aware of the input structure (\textit{smart fuzzers}) or not (\textit{dumb fuzzers})
    \item whether they are aware of the program structure (\textit{white-box}), partially aware (\textit{gray-box}) or not (\textit{black-box})
\end{itemize}
It's important to mention that there is a trade-off between results, time spent testing and resources available.
\newline
Fuzzing is a technique that is quite resource-intensive, putting the CPU cores under heavy load while also generating potentially enormous log files, and these tests are usually run for several hours or even days, but preparing the right set of inputs that will be used to perform the tests is crucial to obtain good results.
\newline
Moreover, during a fuzzing session, there will be a point in time where either the inputs generated by the fuzzer will not trigger new execution flows or no new bugs are discovered for quite some time: reaching this state does not mean that the program is finally free of other bugs, as the conditions to trigger them might simply have not happened yet, and it's obviously impossible to know when this will happen as bugs may be discovered randomly.
\newline \newline \newline
Among the many results achieved thanks to this technique, two honorable mentions can be linked to the popular fuzzer \textit{American Fuzzy Lop} (also known as \textit{AFL} \cite{AFL}).
\newline
In September 2014, AFL was used to discover "Shellshock" \cite{shellshock} (also known as "Bashdoor"), a family of security bugs that affected the Unix Bash shell allowing malicious users to execute arbitrary commands without confirmation.
\newline
In April 2015, AFL discovered the famous "Heartbleed" \cite{heartbleed} bug in OpenSSL, which allowed malicious users to decipher the otherwise encrypted communication from the TLS protocol. 
\newline \newline
This thesis will focus on two campaigns maintained by Google called \textit{OSS-Fuzz} \cite{doc-ossfuzz} and \textit{FuzzBench} \cite{{doc-fuzzbench}}: the first is a free platform that allows open-source developers to fuzz their programs autonomously relying on the computing resources provided by the Google Cloud service, while the latter allows fuzzer developers to test and improve their tools on real-word benchmarks thanks to automated tests and periodic reports.
\newline \newline
More specifically, it revolves around fuzzing some of the projects that have been implemented in these repositories using alternative approaches, trying to discover new and relevant bugs that will be then securely reported and disclosed to their developers in hope to have them fixed.

\newpage
\matteo{Don't use random newpages and newlines; let sapthesis do its things. Use double newlines (intendo il tasto invio, non il comando) instead of the command newline, or are you using it for some specific reason?}
\section{What is Fuzz Testing}
\matteo{I'd call this section "Fuzz Testing" and I would merge 2.2 as a paragraph of this section (or maybe a subsection, you choose).}
As previously mentioned, \textit{fuzz testing} is a software testing technique that feeds invalid, random and unexpected data to a program with the objective of discovering inputs whose execution may lead to crashes, failing assertions and memory leaks.
\newline \newline
We mainly distinguish between 3 types of fuzz testing:
\begin{itemize}
    \item \textbf{application fuzzing:} used for UI elements (such as buttons, input fields) or command-line programs, tests may include high-frequency inputs, providing random/invalid content and inputs exceeding the expected size
    \item \textbf{protocol fuzzing:} used to test the behavior of network elements and servers when invalid messages are sent over a chosen protocol, useful to ensure that such content is not misinterpreted and potentially executed as commands
    \item  \textbf{file format fuzzing:} used for programs that accept "structured inputs", meaning files that have a precise and standard format (like .doc, .jpg) which the fuzzer can modify with the intention of triggering unwanted behavior
\end{itemize}
This work will focus on testing programs that accept input files to provide their services, and for this reason several corpora were used to instruct the fuzzer on the appropriate file format. This will be discussed more in depth in section \ref{fuzzers}.
\newline \newline
One of the main requirements in fuzz testing is to achieve a high degree of \textbf{code coverage}: this statistic measures the percentage of source code executed in a test session, therefore achieving a high value means lowering the chance of leaving parts of the program with undetected bugs. Given this, one could argue that the fuzzer should be aware of the program's structure to maximize the coverage, but given that this process requires a non-trivial overhead it comes with its own trade-offs.
\newline \newline
For this reason, we define 3 different approaches.
\newline \newline
The \textit{black-box testing} involves using a fuzzer that is completely unaware of the program's structure, therefore assumes the program as a simple machine that takes a random input and generates an according output. This approach is relatively fast, can be easily parallelized and has good scalability. On the other hand, it will most likely find only "surface" bugs, i.e. they do not require particular conditions to be met to be triggered.     
\newline \newline
The \textit{white-box testing} involves using a fuzzer that employs "program analysis" techniques to systematically explore and reach critical program locations through meticulously crafted inputs, allowing you to discover bugs that could be potentially hidden deep in the program, although heavily relying on this knowledge implies that bugs related to unknown aspects of the program can be easily missed. While this approach is arguably the most effective one, the time used to analyze the program as well as generating such specialized input exponentially increases with the program complexity.
\newline \newline
The \textit{gray-box testing} attempts to integrate the best aspects of both approaches: use a minimal amount of knowledge over the program's structure to achieve a sufficient degree of code coverage such that the results obtained are satisfactory. This is usually done thanks to "instrumentation", discussed in section \ref{sanitizers}.
The previously mentioned AFL, which has been extensively used in this work, falls in this category.

\newpage
A fuzzing session usually yields two outputs: a set of "interesting" inputs and a list of potential bugs.
\newline \newline
Whenever a generated input results in a new execution flow, it is saved along with others "interesting inputs", which can be used in subsequent fuzzing sessions to provide the fuzzer with even more information about the structure of a good input that explores the deepest parts of the program. Then, the developer might decide to analyze and group these lists, to create a new one that does not contain duplicates and/or inputs triggering the same execution flows, which is particularly useful to ensure that the size does not explode over time.
\newline \newline
Regarding the discovering of bugs, a fuzzer has to be sensible enough to distinguish between crashing and non-crashing inputs without having full knowledge over the program tested, and "sanitizers" are used to inject assertions that make the program crash when a particular kind of failure is detected. Section \ref{sanitizers} explains this concept more thoroughly.
\newline
Then, the fuzzer might produce one (or more) inputs that triggered different kind of bugs, and the developer has to perform what is called \textit{bug triage}:
\begin{itemize}
    \item execute each input individually and observe the output
    \item determine which kind of error occurred and why, maybe also introducing debugging tools
    \item fix the bug entirely, if possible, or at least patch the problem as much as possible
    \item ensure that the bug does not occur in future fuzzing sessions by including the triggering input(s) in the set used for the tests
\end{itemize}
\ \\
Finally, it's important to recall that running a fuzzing session for several days or weeks without bugs does not necessarily imply that the program is secure: this process is driven by randomness, initial inputs provided and environment used, meaning also that each fuzzing session will always result in slightly different results and coverage achieved. Yet, testing a program on all possible inputs is obviously impossible.
\newline
Moreover, it easily generates false positives (trivial or benign warnings) and false negative results (incorrect or misleading results), further overloading the already tedious work of manually assessing each bug.

\ \\ \newline \newline
\newline \newline
Explain the main shortcomings of fuzz testing (maybe also available solution to mitigate them???)...



\newpage
\section{Different types of Fuzz Testers} \label{fuzzers}
A fuzzer is composed by the following key elements. \cite{afl_docs}
\newline \newline
\textbf{Observer.}\ \ \ Provides information observed during the execution of a program to the fuzzer. Such information may be relatively simple, like the total running time for a single test and its output, to more advanced ones, like the maximum depth of the stack in a single execution. They are usually not preserved across many execution, unless an "interesting input" is encountered, in which case they are relayed to other nodes to improve future fuzzing sessions. 
\newline \newline
\textbf{Executor.}\ \ \ Responsible for defining how the program will be executed and the arguments passed on each run. The input for a single test is provided either by writing it in some specific memory location or passed as argument to a so called "harness function", although each fuzzer has its own implementation of this element. Given this, we briefly mention few standard functionalities that compose this element. The \textit{InProcessExecutor} runs the "harness function" and provides crash detection. The \textit{ForkServerExecutor} is responsible for spawning different child processes to fuzz. The \textit{TimeoutExecutor} wraps and installs a timeout for another running executor.
\newline \newline
\textbf{Feedback.}\ \ \ Classifies the result of a single execution and determines if the initial input is "interesting" or not, which is usually determined by retrieving information for the observers and analyzing the updated coverage map. It's also possible to define several of these elements, each one with its own objective (crashes, timeout, new execution flow discovered), and combine them in boolean expression to collect more fine-grained results.
\newline \newline
\textbf{Input.}\ \ \ Data taken from an external source and provided to the tested program to observe its behavior, usually in the form of bytes arrays. The first fuzzing session takes a set of inputs that is defined and provided by the developer itself, while future fuzzing session will also rely on previously discovered "interesting inputs". 
\newline \newline
\textbf{Testcase.}\ \ \ Defined as an input and a set of related metadata, like ID, description and expected results.
\newline \newline
\textbf{Corpus.}\ \ \ Location where testcases are stored, usually disk or memory. An example of input corpus may be composed by several testcases with the same properties, like crashing the program under a specific situation. An example of output corpus may be composed by all the testcases that are considered "interesting".
\newline \newline \newline
This thesis focused on using fuzzers capable of generating new structured inputs for the tested program starting from an initial \textit{seed}, specifically files with a precise format, allowing them to distinguish between a valid and an invalid example. 
\newline
However, even files that do not necessarily follow the defined structure can still be used as input for fuzzing, maybe with an even higher chance of triggering unwanted or unexpected behavior: for this reason, an effective fuzzer should be capable of generating inputs that are "valid enough" so that they are not rejected from the program's parser, and "invalid enough" to potentially trigger corner cases.


\newpage
We finally define the two main types of fuzzers.
\newline \newline
The \textbf{mutational-based fuzzers} require a corpus of seed inputs as reference, and generate new inputs by applying "mutators" on the provided seeds: these operations range from flipping single bits or bytes on a given input or performing mathematical operations over them, to adding or removing said information, sometimes even completely randomizing its content. Furthermore, these mutation may be mixed-and-matched into long sequences, to generate even more new inputs.
\newline
A \textit{smart mutational fuzzer} might leverage its knowledge on the input model to switch between different types of inputs, although this information is not always available.
\newline
A \textit{dumb mutational fuzzer}, like AFL, employs random mutations using as reference the content of "interesting inputs", which usually results in a much lower proportion of valid inputs generated.
\newline \newline
The \textbf{generation-based fuzzers} generate new inputs from scratch, usually relying on a good source of randomness to perform this operation, and for this reason they do not depend on the existence of a good corpus nor its quality.
\newline
A \textit{smart generation fuzzer} might leverage the input model provided by the developer to generate valid new inputs, although this information may not always be available.
\newline
A \textit{dumb generation fuzzer} attempts to generate new inputs without any reference, oftentimes putting more stress on the program's parser rather than the program itself, as they generate an overwhelmingly amount of invalid inputs.



\section{Sanitizers}
\matteo{We are missing a section on sanitizers (I just added it). Introduce them, by also highlighting the differences between compiler ones (ASan, MSan, UBSan...) and binary ones (mainly Valgrind and DrMemory). All these things have been presented in papers, be sure to cite them, don't cite their sites if there are papers.}

\matteo{I see you talk about sanitizers later, I would do it here; it also enables you to talk about them when talking about OSS-Fuzz and FuzzBench.}



\newpage
\section{Open-Source Software}
\matteo{I'm not sure this section belongs here in the first place. This is a master's thesis in computer science eng., I'm pretty sure we all know what open source is. Maybe just say something when you talk about OSS-Fuzz?}
The \textbf{Open-Source Software (OSS)} is a computer software developed in a collaborative and public manner, released under a particular license that allows other users to freely use, study, modify and distribute the software and its source code for any purpose: this allows many users to actively participate in the development of a software by proposing changes and new improvements.
\newline \newline
To be eligible as an open-source software, the license's distribution terms must comply with the following criteria: \cite{osd}
\begin{enumerate}
    
    \item \textbf{Free redistribution} \newline 
    The license must not restrict anyone from selling or giving away the software as part of a suite, nor it could be used to require royalties or fees on such sales.
    
    \item \textbf{Source code} \newline
    The program must include its source code, provided in a form that allows other programmers to modify it, as well as a compiled version. If the source code cannot be distributed, it should be easily obtainable thanks to well-publicized means, ideally downloadable from the Internet free of charge.
    It is not allowed to provide obfuscated source code or any partially-compiled form.
    
    \item \textbf{Derived works} \newline
    The license must allow for modifications and publishing of derived works, allowing them to be distributed under the same license of the original software
    
    \item \textbf{Integrity of the author's source code} \newline
    If the developers want to protect the original source code, they must allow the distribution of "patch files" to perform modification of the program at compile time. In this case, the license must explicitly allow distribution of software built from a modified source code as long as any derived works carry a different name or version number with respect to the original software. 
    
    \item \textbf{No discrimination against person or groups} \newline
    The license must not discriminate against any person or group of persons.
    
    \item \textbf{No discrimination against fields of endeavor} \newline
    The license must not restrict anyone from using the program in a particular field of endeavor or work.
    
    \item \textbf{Distribution of license} \newline
    The license's rights provided must apply for anyone that obtains the product, whether it is the original software or a redistributed version of it.
    
    \item \textbf{License must no be specific to a product} \newline
    The license's rights must not depend on the program being part of a suite. If that is true, the license of such suite must follow the same rights as those granted with the original software distribution. 
    
    \item \textbf{License must not restrict other software} \newline
    The license must not put restrictions of any other software that might be distributed along with the licensed software.
    
    \item \textbf{License must be technology-neutral} \newline
    No license provision may be linked to a particular technology or interface style.
\end{enumerate}
There are some key points that should be considered during the development of open-source software.
\newline \newline
First, the authors must decide how the program will be developed.
\newline
Usually, this software is released under two development branches: a "stable" version, composed by all the functionalities that have been thoroughly tested and work as intended, and a "build" version, that is slightly buggier as it includes proposed changes and new features that have yet to be refined.
Releasing the "build" version early not only allows the developers to showcase their work and attract even more new users, but also provides them with feedback from real users that are willing to run untested versions of their software.
\newline \newline
Then, they must decide how they will interact with online users.
\newline
In this sense, providing full access to the source code means that other users can submit new additions to the software, bug reports and code fixes as well as point out mistakes in the documentation, therefore helping the original developers in their works while also improving and refining the product. Moreover, given that each user may have different knowledge and programming skills as well as different testing environments, this allows to test and benchmark the product on a wide range of systems further increasing the probabilities of finding new and unknown bugs that may be specific to a single OS or architecture.
\newline \newline
Finally, it is important to mention that although any user has the rights to mention a bug, error, or mistake in the program, it is still up to the developers to ensure the truthfulness of what has been reported and how to tackle it.
\newline
For example, bugs that are not security-relevant or that may be related to QoL aspects are easily pushed back as secondary problems or simply ignored altogether.
Sometimes, if the developer are kind enough to accept your request but do not have time and resources to solve it, they might ask the user themselves for a proposed fix and cite them in the next patch notes as a way of thanking them.
\newline \newline
As will be discussed in future sections, while many reports produced in this work highlighted several security bugs that have been fixed in short times, some have been ignored due to them being not relevant at the moment of reporting or because they were declared as an incorrect building approach and/or use of the program itself.   
\newline \newline \newline
Given all this, one could argue that providing complete access to the source code and allowing other people to suggest changes poses a real threat to the security of the program.
\newline
History has shown us many times that, given enough time and resources, releasing the source code of a program will result in malicious users discovering bugs and vulnerabilities that could have potentially catastrophic consequences. Moreover, a malicious user might try to suggest a modification in the code that introduces a specific vulnerability or generate a bug report containing false information with attachments that might exploit a previously unknown vulnerability.
\newline
This is why having a large users base is important and one of the main advantages of open-source: if many people are closely watching how the program is being developed, one of them will most likely realize that malicious modifications are being suggested and notify others of the situation.
\newline \newline
Few noticeable mentions (LibreOffice, VLC, Firefox, etc...) ???




\newpage
\section{Fuzzing with Google}
\matteo{we should call this "Continuous Fuzzing" and introduce Google's role in the description.}
\matteo{Also, consider talking about clusterfuzz first and then OSS-Fuzz and Fuzzbench (which you did not include it here, maybe you wrote this before working on fuzzbench). OSS-Fuzz and Fuzzbench rely on clusterfuzz, so it makes sense to talk about clusterfuzz, I guess.}
The \textit{Google Open Source Project} \cite{google_oss} is a campaign started in 2004, one of the oldest open-source campaigns in the industry. 
\newline
It was initially meant to share Google-developed software under open licenses, with the intention of bringing free technology and information sharing to the public, but it quickly became a program dedicated to improving open-source ecosystems as a whole. 
\newline \newline
Thanks to this campaign, many projects became popular and gained worldwide recognition such as Android OS, TensorFlow, the Go programming language, and many more.


\ \\
\subsection{OSS-Fuzz}
The \textit{OSS-Fuzz Project} was created in 2016 after the famous "Heartbleed" vulnerability was discovered in OpenSSl, one of the most popular open-source projects at the time for encrypting web traffic, as a response to provide developers with free fuzzing and private alerts services for their open-source projects.
While it was initially intended for languages that are not memory-safe (C/C++), it is now capable to provide support for other popular languages such as Python, Go, Java and Rust.  
\newline
As of August 2023, it helped identify and fix over 10.000 vulnerabilities and 36.000 bugs across over 1000 projects. \cite{ossfuzz_docs}.
\newline \newline
Projects can be tested using several fuzzing engines (such as LibFuzzer, AFL++, Honggfuzz and Centipede) in combination with Google Sanitizers (ASan, MSan and UBSan), while \textit{ClusterFuzz} acts as the back-end and reporting tool.
\newline \newline
\begin{figure}[h]
\makebox[\textwidth][c]{\includegraphics[width=0.77\paperwidth]{foto/oss-fuzz_architecture.png}}
\caption{OSS-Fuzz main architecture visualized \cite{ossfuzz_docs}}
\label{fig:ossfuzz_architecture}
\end{figure}

\newpage
The process works as follows.
\newline \newline \newline
Initially, the maintainer of an open-source project creates one or more "fuzz targets" that will be integrated with the project's build and test systems. \cite{libfuzzer_docs}
\newline
A "fuzz target" is essentially a function that accepts an input, in this case an array of bytes, and perform some operations with these bytes to test a specific API.
\newline
Although not all projects are expected to implement and maintain their targets in the same way, developers can refer to a guide of recommendations to increase the efficiency and quality of the automated fuzzing tests performed.
\newline
To briefly summarize them, they should:
\begin{itemize}
    \item maintain the source code and targets' build system using some versioning service (like Git)
    \item allow each fuzz target to be compiled with Sanitizers
    \item avoid modular build systems for the fuzz targets (compile all or nothing) and use general-scope compile flags
    \item provide a seed corpus that is regularly updated and extended with new "interesting inputs", also it should have good coverage
    \item provide a dictionary of tokens to instruct the fuzzer on the correct syntax of the inputs, if applicable
    \item periodically check the performances achieved by each fuzz target, meaning their coverage, time spent on execution and solving abrupt errors
\end{itemize}
\ \\
Then, the newly prepared project must be accepted by OSS-Fuzz, which is done by issuing a "pull request" on the project repository with some requested information, such as: project's main repo, language used, building instructions and email addresses to contact on new issues.
\newline
On the other end, a bot periodically checks for new requests and validates their content before accepting/rejecting them.
\newline \newline \newline
Once the project has been accepted as part of the OSS-Fuzz's infrastructure, a "builder" script follows the provided instructions to build the project's fuzz targets and uploads them to a Google Cloud Service Bucket, a file-hosting service.
\newline
This acts as a middle point between OSS-Fuzz and ClusterFuzz, which uses the aforementioned bucket to download all the necessary elements to fuzz the project as well as upload the results achieved.
\newline \newline \newline
After a successful fuzzing session, any bug discovered is reported to the OSS-Fuzz issue tracker \cite{ossfuzz_bugtracker}, which uses the metadata sent by ClusterFuzz to automatically create a report. 
\newline
Developers have 3 ways of dealing with this situation: they can commit new changes to fix the bug (verified by ClusterFuzz before closing an issue), assign the tag "WontFix" to the bug to notify that it will not be solved, or simply ignore it altoghether. 
\newline \newline
These steps create a cycle that allows for continuous fuzzing and improvement of the software.


\newpage
Finally, OSS-Fuzz follows a strict \textit{bug disclosure guideline}. \cite{bug_disclosure}
\newline \newline
When a bug is discovered, an automatic email is generated and sent to all email addresses specified in the project, and an issue is opened on the issue tracker.
\newline
This email contains the report created using ClusterFuzz, as well as an estimation of the priority and severity of the bug discovered.
\newline \newline
From this moment, the issue will be publicly visible in 90 days or after the fix is released (whichever comes earlier), meaning that anyone will have access to the causing input as well as any other debugging information related to what happened and how to reproduce the bug.
\newline
Before the deadline expires, the developers may request a 14-day grace period if the patch is set to be released on a specific day within this extended period, in which case the public disclosure is delayed.
\newline \newline
In any case, Google reserves the right to change deadlines forwards or backwards depending on the circumstances and the severity of the findings.





\ \\
\subsection{ClusterFuzz}
The \textit{ClusterFuzz Project} is a scalable fuzzing infrastructure with the objective of discovering security and stability issues in software, it is the main platform used by Google to test its own products and also the fuzzing back-end for \textit{OSS-Fuzz}.
\newline
As of May 2023, it discovered over 25.000 bugs in Google proprietary software (e.g. Chrome) and 36.000 bugs with OSS-Fuzz. \cite{clusterfuzz_docs}
\newline \newline
It is based on a highly scalable distributed system of VMs, performing fully automatic bug filing, triage and closing as well as performance reports.
\newline \newline
\begin{figure}[h]
\makebox[\textwidth][c]{\includegraphics[width=0.665\paperwidth]{foto/clusterfuzz_architecture.png}}
\caption{ClusterFuzz main architecture visualized \cite{clusterfuzz_docs}}
\label{fig:clusterfuzz_architecture}
\end{figure}
\ \\
All operations are performed by two components.
\newline \newline
The \textit{App Engine} provides a web interface to the information collected during each fuzzing session, allowing the developers to easily access crashes, results and other information. This is also where tests can be scheduled, which is done via \verb|cron| jobs.
\newline \newline
The \textit{Fuzzing Bots Pool} is a cluster of VMs responsible for running the scheduled fuzzing sessions, and they perform the following operations:
\begin{itemize}
    \item \textbf{fuzz:} runs a fuzzing session
    \item \textbf{progression:} checks if a testcase still reproduces or if has been fixed
    \item \textbf{regression:} calculates the revision range in which a crash was introduced
    \item \textbf{minimize:} eliminates duplicate testcases from the input seeds
    \item \textbf{pruning:} minimize a corpus to the smallest size based on coverage information
    \item \textbf{analyze:} runs a manually uploaded testcase against a specific job to see if it crashes
\end{itemize}
\ \\
Given that some of this tasks are critical and should be treated as atomic operations, bots can be \textit{preemptible} or \textit{non-preemptible}.
\newline
The first refers to a machine that can only run the "fuzz" task as it can be shut down at any moment.
\newline
The latter refers to a machine that is not expected to abruptly stop or crash, therefore is capable of running all tasks.
\newline \newline \newline
Each VMs performs these operations inside Docker instances, created and provided by the developer using Dockerfiles, that are configured with all the tools and files necessary to correctly build and launch the fuzzing targets.





\ \\
\subsection{FuzzBench}
The \textit{FuzzBench Project} is a free service that provides fuzzers' developers with several real-world benchmark tested at Google scale, comparing the results with other famous fuzzers (such as AFL and LibFuzzer) and allowing them to evaluate their performances thanks to daily reports for further improvements.
\cite{fuzzbench_docs}
\newline
\begin{figure}[h]
\makebox[\textwidth][c]{\includegraphics[width=0.67\paperwidth]{foto/fuzzbench_architecture.png}}
\caption{FuzzBench main architecture visualized \cite{fuzzbench_docs}}
\label{fig:fuzzbench_architecture}
\end{figure}
\ \\
The process works as follows.
\newline \newline
Initially, a fuzzer developer integrates its product within FuzzBench using a Dockerfile, containing all the resources necessary to build targets using the fuzzer and where all benchmarks will be executed.
\newline
Similarly to OSS-Fuzz, this process is done via "pull requests", which are automatically revisioned and accepted by bots.
\newline \newline
Then, the developers may choose between two testing approaches: standard and OSS-Fuzz.
\newline
In the first case, the benchmark is created and defined by the developers themselves, and this requires the definition of fuzz targets, build files and Docker images that will be used to correctly build and link the fuzzer to the targets and run the tests.
\newline
In the latter, the developers employ a fuzz target from any OSS-Fuzz project as benchmark, allowing them to test their product on a real-world scenario.
\newline \newline
Finally, a report will be created highlighting the strengths and weaknesses of the fuzzer on the various benchmarks, comparing individual and overall results with other fuzzers.



\chapter{Methodology} \label{chap_3}
\ \\

The aim of this thesis was to analyze common automated testing campaigns, specifically OSS-Fuzz and FuzzBench as they are among the most popular ones, to see if there were any overlooked bugs in the projects that were integrated: we define an overlooked bug as a bug for which a triggering input is produced by the fuzzing framework, but that is not successfully reported as such. The two frameworks considered provide different opportunities for bugs to be overlooked, as OSS-Fuzz uses an automated approach to how tests are scheduled, executed, and how bugs are reported, while FuzzBench requires manual scheduling and bug reporting: this, in turn, meant studying their workflow and examining how programs are tested, to verify how efficiently such frameworks are used by open-source developers.
As previously mentioned, all relevant bugs were then appropriately reported.

In OSS-Fuzz, where fuzzing is performed automatically, we looked for cases where machines have failed: this could happen either due to a fault in the automation process or due to the integration choices made by the developers. In the former case, a fuzzer may succeed in producing a testcase that triggers a bug, but due to some problems in the implementation of the testing environment it is not successfully reported as such. In the latter, it is strictly related to the content of the \textit{project.yaml} configuration file: different sanitizers look for different types of bugs, while each fuzzing engine uses its own set of strategies to test a program, producing results that may be completely different from the other provided fuzzers. Given this, OSS-Fuzz was analyzed by taking a small selection of projects and testing them locally on the latest version with a fixed combination of fuzzing engine and sanitizers, using the latest public fuzzing queue available on the project's repository as input corpus. 

In FuzzBench, where projects are actively tested by several different fuzzers, we looked for bugs discovered in older versions that were not tested on newer ones, which obviously is a human error. As previously mentioned, this framework is based on a small selection of older (and bugged) versions of OSS-Fuzz projects that are continuously tested by several different fuzzers using a predefined testing environment, producing daily reports available both to the fuzzers' developers and the developers of the selected projects, along with public access to the standard corpora used to perform the tests, the results and the crashes found (if any). It is then responsibility of the project developers to analyze and fix these bugs, although this does not always happen. Given this, FuzzBench was analyzed by testing all available benchmarks and the ones from the SBFT '23 conference (see \ref{conference}), building the projects to their latest OSS-Fuzz version using AFL as fuzzing engine, ASan+UBSan as sanitizers, and using as input corpus all the available crashes for each benchmark found in all fuzzing session that were publicly accessible between 2020 and October 2024 (at the moment of testing).


\section{Setting up the environment}
All tests were performed on two separate machines with similar specs, both equipped with personal installations of Ubuntu 22.04 LTS already run-in and used, employing the following tools:
\begin{itemize}
    \item \textit{Docker} \cite{Docker}: virtualization tool used to run the different containers needed by OSS-Fuzz to create the environments where each project was built and tested
    \item \textit{Valgrind} \cite{Valgrind_1}\cite{Valgrind_2}: dynamic binary instrumentation (DBI) framework with tools that perform analysis, profiling, and management of a program during its execution, used specifically for the "Memcheck" tool when looking for memory-related bugs in fuzz targets built without sanitizers
    \item \textit{Python} (v. 3.10.12) \cite{python}: needed by OSS-Fuzz to provide its functionalities and used to create several scripts to perform information and web scraping, reports analysis and bug deduplication
    \item \textit{gsutil} \cite{gsutil}: command-line tool suite provided by Google to remotely access data stored on Google Cloud Service from your local machine, used to analyze FuzzBench experiments and download other resources
    \item \textit{Google Chrome/Development Driver} \cite{driver}: used during the information scraping phase of this work, discussed in section \ref{selection}
\end{itemize}

Regarding \textit{OSS-Fuzz}, most of the preparation was done using its GitHub repository, which was cloned locally on both machines. To provide its services, OSS-Fuzz uses several Python scripts that can be invoked from the command line with appropriate arguments: these commands can be used to update the repository and the files used to build each project, update the base images and each project image, build the project image and its fuzzers, and finally download other resources such as reports and publicly available corpora. 

Regarding \textit{FuzzBench}, most of the preparation was done by incorporating the \textit{gsutil} suite in a Python script that performed web scraping and downloaded content from its Google Cloud bucket, specifically from the location where all experiments results are collected.
\ \\

All tests were run by providing Docker with unlimited access to the CPU cores and disk space, while RAM was limited to 16GB, to ensure that no input may cause an out-of-memory situation that could result in a complete crash of the testing machines. 


\newpage
\section{OSS-Fuzz}
\subsection{Selecting the projects} \label{selection}
At the time of writing, the OSS-Fuzz campaign includes over 1000 projects that are actively being fuzzed and tested, but rebuilding and testing all of them locally would require not only extensive knowledge of many programming languages, but also time and skills to fix potential build errors left by developers' negligence. For this reason, this work focused exclusively on projects written in C/C++, as these languages are widely known to be prone to human error. Then, to further narrow down the analysis, I identified 5 different categories based on the number of sanitizers selected by the developers for testing, keeping ASan as the reference due to its popularity and efficacy. Finally, I ordered each set by "highest number of bugs issued" using the OSS-Fuzz bug tracker and tested these lists from top to bottom until I had 5 projects for each category that were building and fuzzing correctly.

\begin{figure}[h]
\centering
\includegraphics[scale=0.5]{foto/project_yaml.png}
\caption{Example of content from a project.yaml}
\label{fig:project_yaml}
\end{figure}

The first step in this process was to extract the list of all projects written in C/C++ and divide them according to the sanitizers used, and this was done by performing a preliminary analysis of the \textit{project.yaml} configuration file present inside each project's directory. To retrieve this information, I wrote a simple Python script that iteratively explored each project's directory looking for the aforementioned configuration file, opened the file (assuming it was found) and scanned each line looking for the "language: c" string and the keywords "address", "memory" and "undefined", eventually saving the name of each project in the appropriate list.
\newline \newline
This yielded a total of 524 projects out of 1277 written using C/C++, with: 
\begin{itemize}
    \item 238 projects using all sanitizers when fuzzing
    \item 22 projects using ASan+MSan when fuzzing
    \item 62 projects using ASan+UBSan when fuzzing
    \item 46 projects using only ASan when fuzzing
    \item 156 projects not using any sanitizer when fuzzing
\end{itemize}

The next step was to determine the number of bugs already found for each project, and this required a thorough analysis of the "OSS-Fuzz Issue Tracker" website \cite{ossfuzz_bugtracker}. At the time of writing, the issue tracker platform has changed from "Monorail" to "Google Sites", so most of the work described here may no longer work as intended.

\begin{figure}[h]
\makebox[\textwidth][c]{\includegraphics[width=0.67\paperwidth]{foto/issue.png}}
\caption{Example of bug report \cite{ossfuzz_bugtracker}}
\label{fig:issue}
\end{figure}

Since Monorail APIs could only be accessed by the developers of projects already integrated into OSS-Fuzz and there were no files that could be used for offline analysis, I had to perform web scraping on the individual issues from their web reports. To do this, I used as reference a GitHub repository written by Zhen Yu Ding called "Monorail Scraper" \cite{scraper}, a Python script for scraping and retrieving data from Monorail-based platforms, which also included functionalities for ClusterFuzz-generated OSS-Fuzz issues. The tool relies on Google Chrome and its development tool called "ChromeDriver" \cite{driver}, an autonomous web server implementing the "W3C WebDriver" \cite{driver_standard}, which in turn provides a remote interface to control user-agents and a set of interfaces to perform analysis and manipulation of DOM elements. Essentially, all these components combined allow the user to write scripts that, in turn, instruct the Google Chrome browser to visit a particular web page, analyze its DOM elements, and possibly perform some interaction with it.


\newpage
The produced Python web-scraping script requests a range of "report IDs" to be retrieved, then opens a new Google Chrome instance and performs a connection to a specifically constructed link on the Monorail website, attempting to reconnect only once if the first fails. If the resulting DOM displays a login form, it means that the requested bug is still in the disclosure window, in which case the next ID is analyzed. Assuming that the requested report is publicly available, the DOM is scanned for key information, finally stored in JSON files. 
\newline

This analysis was performed on all bugs between 2023-01-01 and 2024-06-31, for a total of 11743 collected reports.

\begin{figure}[h]
\makebox[\textwidth][c]{\includegraphics[width=0.7\paperwidth]{foto/json.png}}
\caption{Example of information collected from a report}
\label{fig:report}
\end{figure}


The second to last step was to analyze the JSON files and make a list of the most buggy projects for each category, which was done by writing a simple Python script that takes as input these files and analyzes the information fields collected, shown in Figure \ref{fig:report}. First, it checks the \textit{"metadata"} field for values such as "WontFix", "Duplicate" or "Invalid": the first means that the developers themselves tagged that specific bug as non-relevant and therefore will not be addressed in the future, the second refers to a report for a bug that has been already issued but was triggered by a different testcase, while the last one means that the reported bug could not be reliably reproduced using the provided testcase. Then, it checks the \textit{"description"} field for manual reports, which have been ignored as the information they contained was not always written according to the same standard used by ClusterFuzz and therefore not useful towards the collection of the information required. Assuming the report analyzed is valid and generated by ClusterFuzz, it retrieved the project name from the \textit{"oss\_fuzz\_bug\_report"} fields, using a dictionary key-value to keep track of the number of bugs reported for each project. 

\newpage
The final step was to analyze all the bugs reported by each project individually and determine which fuzz target produced the highest number of reports. Given the previous script, I extended it to take the project name as input, so that the parsing of the JSON file focused only on reports for that particular project, and the dictionary key-value was now used to keep track of the number of bugs produced by each fuzzing target binary. 
\newline

All the aforementioned work produced Table \ref{fuzzing-table}, which shows all the selected projects and their respective most bugged binaries tested:
\begin{table}[h!]
\makebox[\textwidth]{\begin{tabular}{|l|l|l|}
\hline
\textbf{Project} & \textbf{Sanitizer} & \textbf{Tested binary} \\ 
\hline
binutils         & ALL                 & fuzz\_objdump\_safe            \\
harfbuzz         & ALL                 & hb-subset-fuzzer               \\
imagemagick      & ALL                 & encoder\_heic\_fuzzer          \\
libxml2          & ALL                 & valid                          \\
skia             & ALL                 & skruntimeeffect                \\ 
\hline
gpsd             & A+M                 & FuzzPacket                     \\
libyang          & A+M                 & lyd\_parse\_mem\_json          \\
llvm             & A+M                 & clang-fuzzer                   \\
openjpeg         & A+M                 & opj\_decompress\_fuzz\_J2K     \\
wasmedge         & A+M                 & wasmedge-fuzztool              \\
\hline
cairo            & A+UB                & svg-render-fuzzer              \\
clamav           & A+UB                & clamav\_dbload\_YARA\_fuzzer   \\
freerdp          & A+UB                & TestFuzzCoreClient             \\
tarantool        & A+UB                & luaL\_loadbuffer\_fuzzer       \\
vlc              & A+UB                & vlc-demux-dec-libfuzzer        \\ 
\hline
fwupd            & ASan only                & uswid\_fuzzer             \\
glslang          & ASan only                & compile\_fuzzer           \\
inchi            & ASan only                & inchi\_input\_fuzzer      \\
radare2          & ASan only                & ia\_fuzz                  \\
zeek             & ASan only                & zeek-ftp-fuzzer           \\ 
\hline
fluent-bit       & NONE                & flb-it-fuzz-cmetric\_decode\_fuzz\_OSSFUZZ         \\
gpac             & NONE                & fuzz\_probe\_analyze                               \\
libdwarf         & NONE                & fuzz\_debug\_str                                   \\
libredwg         & NONE                & llvmfuzz                                           \\
serenity         & NONE                & FuzzJs                                             \\
\hline
\end{tabular}}
\vspace{10pt}
\caption{Open-source projects tested and fuzzing binaries analyzed}
\label{fuzzing-table}
\end{table}





\newpage
\subsection{Testing with OSS-Fuzz} \label{test}
The OSS-Fuzz repository contains several Python scripts to build and test the available projects, as well as debugging and reproducing crashes and bugs. Most of the tools used in this work are provided by the "helper.py" script, which I used to download a project's Docker image, build the fuzzers and download the latest public corpora made available by the developers, using the following commands:
\begin{verbatim}
    $ python3 helper.py pull_images 

    $ python3 helper.py build_image {project_name}

    $ python3 helper.py build_fuzzers {project_name}
        --sanitizer={address(default),memory,undefined,none} 
            --engine={libfuzzer(default),afl, honggfuzz, centipede}
        
    $ python3 helper.py download_corpora 
        --project={project_name} --fuzz-target={binary_name}
\end{verbatim}

The \verb|pull_images| argument connects to OSS-Fuzz's Google Bucket to download and update all the Docker "base images" on your local machine, which are needed by all projects to create their testing environments.
The \verb|build_image| argument takes as input the name of a project and builds its Dockerfile, creating a new image on your local machine that will be later used to perform fuzzing. During this process, all dependencies and resources needed to correctly compile the fuzzers are downloaded and installed, including the main \textit{build.sh} script. It was used in conjunction with the previous command and daily \verb|git pull| on the OSS-Fuzz repository, to make sure that I was always building with the latest versions. 
The \verb|build_fuzzers| argument takes as input the name of a project, a list of possible sanitizers, and a list of possible fuzzing engines to be used during the compilation of the fuzz targets. Although this command accepts only one sanitizer and fuzzing engine at a time, it is possible to mix them by acting on some environment variables provided by the base images. This command also acts as a "wrapper" for a much more complex Docker command, that loads the project's Docker image using some specific environment variables and invokes the execution of the \textit{build.sh} script.
The \verb|download_corpora| argument takes as input a project name and a fuzz target, it then connects to the project's Google Bucket and downloads the latest public corpus for the provided fuzz target.

After executing these commands, three new directories are created: \verb|out| contains the project's directory where all built files were saved, including libraries, fuzz targets and other files created by the selected fuzzer, \verb|work| acts as a temporary location to store intermediate files during the building process and the fuzzing sessions, and \verb|corpus| contains the downloaded corpora stored as a zip file.

To prepare the tests, I was tasked with building the chosen fuzz targets using AFL++ (current state-of-the-art fuzzer) and with all possible sanitizers, meaning that each project was compiled 4 times: with ASan only, with MSan only, with UBSan only, and without any sanitizer.

\newpage
The tests were performed inside each project's Docker image, created and configured using the following command:
\begin{verbatim}
    $ docker run --rm --privileged 
        --platform linux/amd64 --memory=16g 
        -v /oss-fuzz/build/out/{project_name}/:/out/
        -v /oss-fuzz/build/corpus/{project_name}/:/corpus/    
        -v /home/zio-saba/Scrivania/TESI/logfiles/:/logfiles/ 
        -it  gcr.io/oss-fuzz/{project_name} /bin/bash
\end{verbatim}
The first parameters are needed to create a privileged instance of Docker, specify the running platform on which the fuzz targets will be tested and limit memory usage inside the container. The arguments starting with \verb|-v| are used to create a shared directory, i.e. linking a local directory to a virtual one created inside the container: this was necessary to make sure that I could access the resources stored locally on my machine (i.e. fuzz targets, libraries and their corpus) from inside the Docker container. The last line invokes the project image to load as well as making it interactive by spawning a \verb|/bin/bash| process.
\ \\

Once the Docker image has been loaded and ready to use, some final adjustments were performed to perform the tests, here we mention a few of them:
\begin{itemize}
    \item all tests performed on fuzz targets built with MSan required the sanitizer's libraries to be copied in some specific locations, using the following commands:
\begin{verbatim}
$ cp -R /usr/msan/lib/* /usr/local/lib/x86_64-unknown-linux-gnu/
$ cp -R /usr/msan/include/* /usr/local/include
\end{verbatim}
    \item all tests performed on fuzz targets built without sanitizers relied on Valgrind to perform binary analysis and profiling, installed using the following command:
\begin{verbatim}
$ sudo apt install gdb valgrind
\end{verbatim} 
    \item when modifying the source files for compile errors, the \verb|compile| command was used to start the build process for the fuzz targets without leaving the Docker image
    \item sometimes, the \verb|apt| tool was not available in a specific Docker container, because all base images provided by OSS-Fuzz contain a minimal installation with only some key packages that are usually enough to compile programs, such as compiler, assembler, text editors and standard libraries: in such cases, the \verb|unminimize| command was executed, which essentially "unpacks" the container and reverts it to a standard Ubuntu image, reinstalling all the default packages as well as standard additional tools
\end{itemize}


\newpage
When testing fuzz targets built with ASan, MSan or UBSan, I used the following command:
\begin{verbatim}
    $ for i in /corpus/*; do 
        echo "TEST" $i; 
        echo "TEST" $i >> /logfiles/PROJECT_SANITIZER-NAME.log; 
        ./{fuzz_target} $i &>> /logfiles/PROJECT_SANITIZER-NAME.log; 
      done
\end{verbatim}
The shell \verb|for| construct scans all files found in the "corpus" directory, then prints the name of the current testcase on the terminal for monitoring purposes, and finally logs the name of the current testcase and the results of the fuzz target executed on that particular testcase on a log file.

\begin{figure}[h]
\makebox[\textwidth][c]{\includegraphics[width=0.8\paperwidth]{foto/ubsan_example.png}}
\caption{Example of integer-overflow bug reported by UBSan}
\label{fig:ubsan_example}
\end{figure}


\newpage
When testing fuzz targets built without any sanitizers, I used Valgrind to analyze and profile the binary's execution, using the following command:
\begin{verbatim}
    $ for i in /corpus/*; do 
        echo "TEST" $i; 
        valgrind --log-fd=9 9>>/logfiles/PROJECT_valgrind.log 
            ./{fuzz_target} $i >> /logfiles/PROJECT_valgrind.log 
            && echo -e "\n\n" >> /logfiles/PROJECT_valgrind.log; 
      done
\end{verbatim}
Similarly to before, the shell \verb|for| construct scans all files found in the "corpus" directory and prints the name of the current testcase on the terminal for monitoring purposes, then Valgrind is invoked with a custom file descriptor to redirect \verb|STDERR| in the log file, and the name of the current testcase and the results of the fuzz target executed on that particular testcase are logged on a log file.

\begin{figure}[h]
\makebox[\textwidth][c]{\includegraphics[width=0.65\paperwidth]{foto/valgrind_example.png}}
\caption{Example of use-of-uninitialized-memory (UUM) bug reported by Valgrind}
\label{fig:valgrind_example}
\end{figure}


\newpage
However, not all projects successfully built on their first attempt.
In most cases, the building process failed due to missing libraries: for example, libraries like \verb|pthread| and \verb|math| were often automatically added by the sanitizers. Other times, libraries were not being correctly linked and/or were missing crucial compiling flags, like \verb|-ldl| and \verb|-lz|. To fix these problems, I had to manually modify the source files and compile the fuzz targets from inside the Docker container, using the following command:
\begin{verbatim}
    $ docker run --rm --privileged 
        --platform linux/amd64 --memory=16g
        -e PROJECT_NAME={project_name} -e HELPER=True 
        -e FUZZING_LANGUAGE=c++ 
        -e FUZZING_ENGINE=afl 
        -e SANITIZER={address,memory,undefined,none} 
        -v /oss-fuzz/build/out/{project_name}/:/out/   
        -v /oss-fuzz/build/work/{project_name}/:/work/
        -it  gcr.io/oss-fuzz/{project_name} /bin/bash
\end{verbatim}
Similarly to before, the first parameters are needed to create a privileged instance of Docker, specify the running platform on which the fuzz targets will be tested and limit memory usage, the arguments starting with \verb|-v| are used to create a shared directory in the Docker container and the last line invokes the project image to load as well as making it interactive by spawning a \verb|/bin/bash| process.
The novelty lies in the arguments starting with \verb|-e|, which can be used to override the environment variables provided by the Docker image: these variables can be used by the source files to compile a program using the most appropriate compile flags, fuzzing engine and sanitizers, and they can be easily modified by the user when creating the Docker image to easily re-target the compilation with little to none effort. 













\newpage
\section{FuzzBench}
\subsection{Selecting the projects}
The FuzzBench campaign started in 2020 and has been performing tests almost daily for the past 4 years, providing valuable information to both fuzzers developers and the developers of the selected projects that have been integrated as benchmarks.
\begin{figure}[h]
\centering
\includegraphics[scale=0.4]{foto/tree.png}
\caption{Simplified structure of the FuzzBench's Google Cloud directory tree}
\label{fig:tree}
\end{figure}
\ \\
All data on FuzzBench is grouped by test date, and each test set is composed by several key information:
\begin{itemize}
    \item \verb|build-logs|: contains the logs generated when building the fuzz targets
    \item \verb|coverage|: contains information related to the coverage achieved by each fuzzer on the different projects tested that day
    \item \verb|input|: contains the binaries executed to compile the different fuzz targets
    \item \verb|experiment-folders|: contains all the data related to the testing sessions
\end{itemize}

Following the experiments, each project has its own folder that specifies the name of the project, the fuzz target tested, the fuzzing engine used and sometimes also the objective of the session (bugs, coverage, correctness). Each project then undergoes several fuzzing sessions, identified by different \verb|trial| folders, providing all the information that was relevant to this work:
\begin{itemize}
    \item \verb|corpus|: contains several corpora stored as zip files.
    \item \verb|crashes|: contains all the crashes found in each trial as a separate zip file
    \item \verb|results|: contains the cumulative log of all fuzzing sessions
\end{itemize}


The objective of this analysis was to build the latest version of several experiments, download all the available \verb|crashes| for each one of them and verify whether they have been fixed or not. However, a preliminary analysis yielded over 100 million possible zip files to download and test, which would require several months of work by itself. 
As previously mentioned, FuzzBench provides coverage-oriented or bug-oriented fuzzing, therefore I focused on bug-oriented experiments for the analysis. To do this, I found the \textit{experiment-requests.yaml} configuration file \cite{exp_yaml} inside the FuzzBench repository, which contained all information related to the type of tests conducted, the projects tested as well as all the fuzzing engines used:

\begin{figure}[h]
\centering
\includegraphics[scale=0.58]{foto/exp_yaml.png}
\caption{Excerpt from the experiment configuration file}
\label{fig:exp_yaml}
\end{figure}

Similarly to the process shown in section \ref{selection}, I wrote a simple Python script taking as input the above file and storing the name of all those experiments containing the string "type: bug". Unfortunately, this file only contained information on all experiments between 2020 and 2023, forcing me to later retrieve all the experiments performed in 2024.

Regarding the selection of the projects and fuzz targets to test, while the initial idea was to perform the analysis only on the benchmarks provided by FuzzBench, it was later decided to also include the special benchmarks added during the "SBFT '23" conference (refer to \ref{conference}) due to the overwhelmingly amount of bugs discovered when performing preliminary tests on them.

Finally, to retrieve all possible crashes from the bug-oriented experiments performed in 2020-2023 and all experiments from 2024, I used the \textit{gsutil} suite provided by Google, which works similarly to the \verb|ls|, \verb|mv| and \verb|cp| command of the Linux shell, except that it takes as argument the URL of a Google Cloud resource. To construct the URL of all the resources needed, I wrote a Python script that essentially performs horizontal scraping over the directory tree shown before (see Figure \ref{fig:tree}), storing at each intermediate step the link of the next resource to list using the \verb|gsutil ls| command. This analysis yielded 971656 zip files, for a total of 9 million crashes to test.
\newline

All the aforementioned work produced Table \ref{fuzzing-table-2}, which shows the standard FuzzBench benchmarks (above) and the SBFT'23 benchmarks (below) along with their respective binaries tested:
\begin{table}[h]
\makebox[\textwidth]{\begin{tabular}{|l|l|}
\hline
\textbf{Project} & \textbf{Tested binary}  \\
\hline
bloaty   &   fuzz\_target                       \\
curl   &   curl\_fuzzer\_http                   \\
freetype2   &   ftfuzzer                        \\
harfbuzz   &   hb-shape-fuzzer                  \\
jsoncpp   &   jsoncpp\_fuzzer                   \\
lcms   &   cms\_transform\_fuzzer               \\
libjpeg-turbo   &   libjpeg\_turbo\_fuzzer      \\
libpcap   &   fuzz\_both                        \\
libpng   &   libpng\_read\_fuzzer               \\
libxml2   &   xml                               \\
mbedtls   &   fuzz\_dtlsclient                  \\
openssl   &   x509                              \\
openthread   &   ot-ip6-send-fuzzer             \\
php   &   php-fuzz-parser                       \\
proj4   &   proj\_crs\_to\_crs\_fuzzer          \\
re2   &   re2\_fuzzer                           \\
sqlite3   &   ossfuzz                           \\
systemd   &   fuzz-link-parser                  \\
vorbis   &   decode\_fuzzer                     \\
woff2   &   convert\_woff2ttf\_fuzzer           \\
zlib   &   zlib\_uncompress\_fuzzer             \\
\hline
arrow   &   parquet-arrow-fuzz                     \\
aspell   &   aspell\_fuzzer                        \\
assimp   &   assimp\_fuzzer                        \\
ffmpeg   &   ffmpeg\_demuxer\_fuzzer               \\
file   &   magic\_fuzzer                           \\
grok   &   grk\_decompress\_fuzzer                 \\
libaom   &   av1\_dec\_fuzzer                      \\
\hline
\end{tabular}}
\vspace{10pt}
\caption{FuzzBench and SBFT'23 benchmarks tested and fuzzing binaries analyzed}
\label{fuzzing-table-2}
\end{table}



\newpage
\subsection{Testing with FuzzBench}
FuzzBench projects were built and tested almost identically to the methodology shown for OSS-Fuzz (see \ref{test}), except for a few small differences.

Initially, I used the "helper.py" script to download the latest Docker image for each project, but the fuzzers had to be built manually: this is because all FuzzBench benchmarks are compiled using a custom combination of sanitize flags from ASan and UBSan \cite{flags}, implying that FuzzBench does not perform memory analysis when fuzzing. To replicate this build configuration, I had to modify the environment variables provided by the Docker images, using the following command:
\begin{verbatim}
    $ docker run --rm --privileged 
        --platform linux/amd64 --memory=16g 
        -e PROJECT_NAME={project_name} -e HELPER=True 
        -e FUZZING_LANGUAGE=c++ -e FUZZING_ENGINE=afl 
        -e SANITIZER=address 
        -e SANITIZER_FLAGS_address="
            -fsanitize=address,array-bounds,bool,builtin,enum,
                integer-divide-by-zero,null,object-size,return,
                returns-nonnull-attribute,shift,
                signed-integer-overflow,
                unsigned-integer-overflow,
                unreachable,vla-bound,vptr
            -fno-sanitize-recover=array-bounds,bool,builtin,enum,
                integer-divide-by-zero,null,object-size,return,
                returns-nonnull-attribute,shift,
                signed-integer-overflow,unreachable,
                vla-bound,vptr 
            -fsanitize-address-use-after-scope" 
        -v /oss-fuzz/build/out/{project_name}/:/out/  
        -v /oss-fuzz/build/work/{project_name}/:/work/
        -v /oss-fuzz/build/corpus/{project_name}/:/corpus/
        -v /home/zio-saba/Scrivania/TESI/logfiles/:/logfiles/  
        -it  gcr.io/oss-fuzz/{project_name} /bin/bash
\end{verbatim}
As previously shown, the first parameters are needed to create a privileged instance of Docker, specify the running platform on which the fuzz targets will be tested and limit memory usage, the arguments starting with \verb|-e| override the environment variables provided by the Docker image, the ones starting with \verb|-v| are used to create shared directories and the last line invokes the project image to load as well as making it interactive by spawning a \verb|/bin/bash| process.
Given that the container accepts only one sanitizer at a time, I had to specify both ASan and UBSan flags in the same variable storing the ASan sanitizer flags, using as reference the subset of UBSan flags used by default by FuzzBench \cite{flags}.

\newpage
All crashes collected during the initial analysis were then categorized as follows:
\begin{itemize}
    \item \textbf{crash:} inputs leading to any scenario that forces the system to close the process, like SEGV and buffer overflows
    \item \textbf{out-of-memory (oom):} inputs inducing memory leaks or using huge allocation sizes to purposely test fail-safe mechanisms
    \item \textbf{timeout:} inputs that are either very long to parse or that purposely introduce unnecessary operations, again to test the resiliency of the program 
\end{itemize}

Finally, tests were performed using the following command:
\begin{verbatim}
    $ for i in /corpus/*; do 
        echo "TEST" $i; 
        echo "TEST" $i >> /logfiles/PROJECT_SANITIZER-NAME.log; 
        timeout 30s ./{fuzz_target} $i 
            &>> /logfiles/PROJECT_SANITIZER-NAME.log; 
      done
\end{verbatim}
As before, the shell \verb|for| construct scans all files found in the "corpus" directory, then prints the name of the current testcase on the terminal for monitoring purposes, and finally logs the name of the current testcase and the results of the fuzz target executed on that particular testcase on a log file. The only addition is the \verb|timeout| command, and that is because due to the presence of inputs that purposely take unnecessary time, I decided to follow the same timeout duration used by ClusterFuzz when fuzzing with AFL++ \cite{timeout}.





\newpage
\section{Bug deduplication} \label{bug_dedup}
To correctly analyze and categorize the bugs discovered, it is necessary to perform \textit{bug deduplication}, a technique used to identify all inputs that produced the same output, which is crucial to remove unnecessary data from the results and make the bug analysis work less tedious for the developer. There are several strategies that can be applied to deduplicate bugs, but it is important to mention that there is no fool-proof methodology that produces perfect results on all possible scenarios, and this is because which information you are using as reference and their number will (almost always) lead to partial loss of data.

We provide an overview of the most popular bug deduplication techniques \cite{dedup_survey}.
The most common technique is \textit{stack trace analysis}, i.e. analyzing input ID, function call records (also called \textit{frames}) and error type: this is because when a program crashes, several debug and error information are collected from the runtime stack, and their analysis has proven to be particularly effective in identifying the uniqueness of a bug.
Another approach is based on \textit{code coverage information analysis}: in the context of fuzzing, the fuzzing tool maintains several bitmaps to store the execution path of each testcase, so that when a bug is encountered, it is considered unique only if the testcase follows a previously unseen execution path or if it matches an already known flow but with a different execution order.
Finally, \textit{context comparison} may be used to differentiate between different execution scenarios: most common methodologies are taint analysis, which involves analyzing how data flows and is manipulated by the program during its execution, and symbolic execution, which performs a static analysis of the code to simulate all possible execution scenarios. However, this last approach is often discarded, as it requires enormous amounts of time, data and knowledge to produce meaningful results.
\newline

Since this work focused primarily on frameworks based on ClusterFuzz as the back-end, I decided to employ the same \textit{stack trace analysis} approach used by ClusterFuzz when generating bug reports, where the fields "Crash Type" and "Crash state" are the main references (see Figure \ref{fig:issue}):
\begin{itemize}
    \item address bugs were deduplicated using as reference the last 3 stack entries and the error type
    \item undefined behavior bugs were deduplicated using as reference the error type, the address where the error occurred and the causing instruction
    \item memory bugs were deduplicated using as reference the last 3 stack entries, the error type and (if available) the signal that caused the spontaneous crash
\end{itemize}

\chapter{Results} \label{chap_4}
\ \\
All collected bugs were appropriately deduplicated according to the methodology described in \ref{bug_dedup}: to do this, I used several Python scripts (one for each sanitizer and one for Valgrind) that analyzed the logs generated during each fuzzing sessions, and deduplication was performed using as reference the error type and the last 3 stack entries from where the error occurred.

Then, \textit{bug triage} was performed, analyzing more in depth where the bug occurred, why it happened and assigning a priority in the range "Low", "Medium" and "High". This required a manual investigation of each bug individually, statically and dynamically (when available), to have a more clear understanding of the problem and potentially provide suggestions to the developers regarding the fix.
This step was also important to refine the previous deduplication step, as two inputs that triggered (apparently) different bugs may only have distinct stack flows: if the bug originated from the same function, it meant that two different execution flows triggered the same error. In this case, a common practice is to fix one of them and then use the other input as an additional check for correctness.

Finally, all recorded bugs for a project, along with their log and fuzz targets, were reported to their respective developers according to their preferred communication method: almost all reports were sent using GitHub's integrated issue tracker, some required a subscription to a proprietary issue tracking platform, and a few of them were sent via emails.

At the moment of writing, unfortunately, not all developers answered and/or acknowledged the reported bugs.



\newpage
\section{OSS-Fuzz}
\begin{table}[h!]
\small
\makebox[0.95\textwidth]{\begin{tabular}{|l|l|l|l|l|l|l|l|l|}
\hline
\textbf{Project} & \textbf{Sanitizers} & \textbf{Queue size} & \textbf{Crashes} & \textbf{ASan} & \textbf{Valgrind} & \textbf{UBSan} & \textbf{Total} & \textbf{Confirmed}  \\ 
\hline
binutils         & ALL                 & $20,274$            & $0$              & $0$           & $1$           & $0$            & $1$             & $1$                 \\
harfbuzz         & ALL                 & $23,357$            & $0$              & $0$           & $1$           & $0$            & $1$             & $1$                 \\
imagemagick      & ALL                 & $9,470$             & $0$              & $0$           & $6$           & $1$            & $7$             & $1$                 \\
libxml2          & ALL                 & $13,474$            & $0$              & $0$           & $0$           & $0$            & $0$             & $0$                 \\
skia             & ALL                 & $18,295$            & $0$              & $0$           & $0$           & $0$            & $0$             & $0$                 \\ 
\hline
gpsd             & A+M       & $5404$  & $0$                 & $0$              & $0$           & $0$           & $0$             & $0$                 \\
libyang          & A+M                 & $8,745$             & $0$              & $0$           & $0$           & $0$            & $0$             & $0$                 \\
openjpeg         & A+M                 & $8,856$             & $0$              & $0$           & $0$           & $0$            & $0$             & $0$                 \\
wasmedge         & A+M                 & $9,454$             & $0$              & $0$           & $0$           & $0$            & $0$             & $0$                 \\
\hline
cairo            & A+UB                & $15,870$            & $0$              & $1$           & $1$           & $28$           & $30$            & $0$                 \\
clamav           & A+UB                & $6,742$             & $0$              & $0$           & $0$           & $2$            & $2$             & $0$                 \\
freerdp          & A+UB                & $7,607$             & $0$              & $0$           & $0$           & $1$            & $1$             & $1$                 \\
tarantool        & A+UB                & $10,987$            & $0$              & $0$           & $1$           & $0$            & $1$             & $1$                 \\
vlc              & A+UB                & $16,018$            & $2$              & $1$           & $2$           & $4$            & $9$             & $5$                 \\ 
\hline
fwupd            & ASan only                & $5,843$             & $0$              & $0$           & $0$           & $0$            & $0$             & $0$                 \\
glslang          & ASan only                & $14,534$            & $0$              & $1$           & $1$           & $0$            & $1$             & $1$                 \\
inchi            & ASan only                & $12,034$            & $0$              & $1$           & $4$           & $3$            & $8$             & $8$                 \\
radare2          & ASan only                & $9,914$             & $0$              & $1$           & $0$           & $9$            & $10$            & $10$                \\
zeek             & ASan only                & $8,390$             & $0$              & $0$           & $1$           & $5$            & $6$             & $6$                 \\ 
\hline
fluent-bit       & NONE                & $4,968$             & $0$              & $0$           & $1$           & $1$            & $2$             & $2$                 \\
gpac             & NONE                & $22,917$            & $2$         & $0$           & $25$          & $0$            & $27$            & $27$                \\
libdwarf         & NONE                & $7,667$             & $0$              & $0$           & $0$           & $0$            & $0$             & $0$                 \\
libredwg         & NONE                & $46,160$            & $0$              & $1$           & $3$           & $0$            & $4$             & $4$                 \\
serenity         & NONE                & $9,940$             & $0$              & $0$           & $1$           & $1$            & $2$             & $2$                 \\
\hline
TOTAL BUGS   &   &   &$4$   &$6$   &$80$   &$56$   &$145$   &$103$       \\
\hline
\end{tabular}}
\vspace{10pt}
\caption{Bugs collected by OSS-Fuzz projects}
\label{ossfuzz-table}
\end{table}
Table \ref{ossfuzz-table} shows the aggregated results for OSS-Fuzz, dividing bugs per category and showing the total number of bugs found compared to the number of bugs that were confirmed by the respective developers after being reported. While it is important to note that this is a very small subset of projects tested (25 out of 1000+) and is not representative of the entire OSS-Fuzz campaign, it is still a good indicator that there are indeed bugs that these frameworks are missing.

\matteo{Yeah, we need to rethink this based on the new experiments we have to do. Since we did not find anything in other MSan + ASan projects we might think of ignoring this category as a whole for the moment.}
This analysis found a total of 145 bugs (an average of 5.8 bugs per project), mostly originated by all projects that were not tested with all sanitizers: this scenario was to be expected, as it would make sense for projects that are being tested as thoroughly as possible to also have less undetected bugs. This is also a good example of how adding a sanitizer to your development cycle is beneficial to the program. However, additional testing is not a replacement for programs written following good coding practices: "Ghostscript" uses MSan yet it has the highest number of memory-related bugs, "Cairo" uses UBSan yet it has the highest number of undefined-behavior bugs. \ziosaba{FIX PREVIOUS PHRASE AFTER NEW PROJECT}

We can also infer the popularity of each sanitizer. ASan is the most widely used, which is reflected by the number of bugs it found. UBSan is also among the most used ones, although the statistic shown here is biased by "Cairo": while this project is affected by many UB bugs, specifically improper conversion of data between different types, almost all previous reports on the issue tracker related to similar problems were simply marked as "Wont Fix". Finally, Valgrind's memory analysis via the "Memcheck" tool yielded the highest number, highlighting how MSan was often not used by developers during tests due to its reports containing too many false positives.



\newpage
\subsection{Case study: GPAC}
The \textit{"GPAC Project on Advanced Content"} (abbreviated \textit{GPAC}) \cite{gpac} is a free and open-source multimedia framework that provides tools to process, inspect, package, stream, media playback and interact with media content, making it a popular choice also among major broadcasters such as Netflix.

Its analysis yielded a total of 27 bugs, 2 crashes and 25 memory-related bugs.
The crashes were related to a SEGV signal sent as a consequence of the process attempting to access address \verb|0x0| (i.e. zero page), which is the first page of a computer's memory and whose access is prohibited and causes an access violation fault, shown below:
\begin{figure}[h]
\makebox[\textwidth][c]{\includegraphics[width=0.55\paperwidth]{foto/segv_gpac.png}}
\caption{Memory access violation reported by Valgrind}
\label{fig:segv_gpac}
\end{figure}
\ \\

It is interesting to note that these bugs have already been reported by ClusterFuzz when discovered by this work and, more importantly, they were still in the bug disclosure windows: this meant that the inputs causing these bugs were not meant to be added to later fuzzing queues and should have not be publicly accessible and available, due to obvious security implications.

The remaining 25 memory bugs were all related to use-of-uninitialized-memory (UUM) and memory-bound errors. More specifically, they can be divided into three categories according to valgrind:
\begin{itemize}
    \item \textit{conditional jump or move depends on uninitialized value}: happens when an \verb|if| statement is based on a variable whose value is uninitialized
    \item \textit{use of uninitialized value in function [...]}: happens when one (or more) of the parameters passed to an invoked function are uninitialized
    \item \textit{invalid read of size [...]}: happens when a read/write operation performed overruns the provided buffer, as the content outside of a buffer could be anything
\end{itemize}

After reporting the UUM bugs and having the pleasure of interacting with the CEO of GPAC itself, we were happy to discover all bugs fixed in just few days. The developers created a public commit \cite{gpac_commit} to explain the issues discovered and reassure everyone that they did not pose a security concern, as they believe the path to exploitation is difficult. However, they agreed on the fact that such bugs could be resolved by trivial and obvious corrections, and the existence of such issues show that they need to be more vigilant: specifically, most of the problems mentioned originated from local structures declared but left uninitialized, and buffer-related operations with no checks regarding the size of the data passed to the buffer as well as recalling that most read/write function return the number of bytes actually read/written.
They concluded by expressing their gratitude, as they have recently been battling with seemingly random bugs causing disruptions with the normal functionality of the program, therefore the report has been valuable in helping them to remove uninitialized variables that were leading to undefined behavior.

This project was a clear example on the importance of using sanitizers during tests, as they highlighted simple and trivial errors that would otherwise be easily overlooked during a manual analysis of the code, but still that greatly helped them in improving the user experience by removing unwanted behavior.








\newpage
\subsection{Case study: VLC}
The \textit{VLC Media Player} (commonly known as \textit{VLC}) \cite{vlc} is a free and open-source media player software and streaming media server developed by the "VideoLAN Project", supporting many audio and video compression methods, file formats and providing many free decoding and encoding libraries.

Its analysis yielded a total of 9 bugs spread across all categories: 2 crashes, 1 heap-buffer overflow, 2 memory-related bugs and 4 undefined-behaviors.
The heap-buffer overflow bug was initially presented as an ASan bug related to an invocation of \verb|memcpy| that was copying a structure into another one of the same type, which initially left the developers a bit perplexed. After a further analysis, they discovered that the input provided was triggering some strange behavior in another function of their program, whose output lead to incorrect usage of the copy function and therefore the buffer overflow.
The crashes were both related to a SEGV signal sent as a consequence of the process attempting to access a low address that was not mapped within the process's memory region, causing an access violation fault. After a further analysis, it was later discovered that both problem originated from a series of read/write operations performed on uninitialized values causing undefined behavior, ultimately leading to the process accessing an erroneous address.
The first memory bug was related to a memory-bound error caused by several reads operation overrunning the provided buffers, leading to undefined behavior. At the moment of writing, the problem has yet to be fixed because the developers are discussing whether to rewrite or remove the source file causing the problem, as they argue that it is very insecure and could pose a threat to the overall security of the product.
The second memory bugs was related to a use-of-uninitialized-memory (UUM) bug when opening a provided input media file. At the moment of writing, a fix has been proposed but it has yet to be merged with the main code.
Finally, the undefined-behavior (UB) bugs were all related to signed integer-overflow errors, which are relatively easy to fix. Yet, at the moment of writing they have not been acknowledged by the developers.

This project was a clear example of the most common behavior encountered during this work when interacting with open-source developers.
When presented with bugs that could potentially lead to vulnerabilities or problems in the normal functionalities of their program, their response is usually quick and the problem is fixed in a short amount of time. 
Instead, when the problems have low or negligible priority (like the undefined-behavior bugs), they either ignore the reports or acknowledge their existence and postpone them indefinitely.



\newpage
\section{FuzzBench}
\begin{table}[h]
\small
\makebox[0.96\textwidth]{\begin{tabular}{|l|l|l|l|l|l|l|l|}
\hline
\textbf{Project} & \textbf{Sanitizers} & \textbf{Queue size} & \textbf{Crashes} & \textbf{ASan} & \textbf{UBSan} & \textbf{Total} & \textbf{Confirmed}  \\ 
\hline
bloaty   &NONE   &$2349$   &$0$   &$0$   &$2$   &$2$   &$2$ \\
curl   &NONE   &$0$   &$0$   &$0$   &$0$   &$0$   &$0$ \\
freetype2   &ALL   &$0$   &$0$   &$0$   &$0$   &$0$   &$0$ \\
harfbuzz   &ALL   &$34634$   &$0$   &$0$   &$0$   &$0$   &$0$ \\
jsoncpp   &A+UB   &$0$   &$0$   &$0$   &$0$   &$0$   &$0$ \\
lcms   &ALL   &$30526$   &$0$   &$0$   &$0$   &$0$   &$0$ \\
libjpeg-turbo   &ALL   &$0$   &$0$   &$0$   &$0$   &$0$   &$0$ \\
libpcap   &ALL   &$111$   &$0$   &$0$   &$0$   &$0$   &$0$ \\
libpng   &ALL   &$0$   &$0$   &$0$   &$0$   &$0$   &$0$ \\
libxml2   &ALL   &$1961857$   &$0$   &$0$   &$0$   &$0$   &$0$ \\
mbedtls   &NONE   &$1$   &$0$   &$0$   &$0$   &$0$   &$0$ \\
openssl   &A+UB   &$0$   &$0$   &$0$   &$0$   &$0$   &$0$ \\
openthread   &A+UB   &$0$   &$0$   &$0$   &$0$   &$0$   &$0$ \\
php   &ALL   &$202235$   &$0$   &$0$   &$0$   &$0$   &$0$ \\
proj4   &NONE   &$208971$   &$0$   &$0$   &$0$   &$0$   &$0$ \\
re2   &ALL   &$0$   &$0$   &$0$   &$0$   &$0$   &$0$ \\
sqlite3   &ALL   &$0$   &$0$   &$0$   &$0$   &$0$   &$0$ \\
systemd   &ALL   &$238072$   &$0$   &$0$   &$0$   &$0$   &$0$ \\
vorbis   &A+M   &$0$   &$0$   &$0$   &$0$   &$0$   &$0$ \\
woff2   &ALL   &$0$   &$0$   &$0$   &$0$   &$0$   &$0$ \\
zlib   &ALL   &$0$   &$0$   &$0$   &$0$   &$0$   &$0$ \\
\hline
arrow   &NONE   &$1163787$   &$0$   &$0$   &$0$   &$0$   &$0$ \\
aspell   &A+UB   &$919701$   &$0$   &$0$   &$2$   &$2$   &$2$ \\
assimp   &NONE   &$312651$   &$43$   &$104$   &$13$   &$160$   &$55$ \\
ffmpeg   &ALL   &$212617$   &$0$   &$0$   &$6$   &$6$   &$0$ \\
file   &A+UB   &$1916360$   &$0$   &$0$   &$0$   &$0$   &$0$ \\
grok   &ALL   &$1492374$   &$0$   &$2$   &$15$   &$17$   &$17$ \\
libaom   &ALL   &$19729$   &$0$   &$0$   &$0$   &$0$   &$0$ \\
\hline
TOTAL BUGS   &   &   &$43$   &$106$   &$38$   &$187$   &$74$          \\
\hline
\end{tabular}}
\vspace{10pt}
\caption{Bugs collected by Fuzzbench and SBFT'23 benchmarks}
\label{fuzzbench-table}
\end{table}
Table \ref{fuzzbench-table} shows the aggregated results for FuzzBench, dividing bugs per category and showing the total number of bugs found compared to the number of bugs that were confirmed by the respective developers after being reported. For the sake of completeness, the table also distinguishes between the standard FuzzBench benchmarks (above) and the "SBFT '23" benchmarks (below), and reports the sanitizers regularly used by the OSS-Fuzz counterpart of the above projects.

It is immediate to see that all bugs came from the "SBFT '23" benchmarks with the exception of "bloaty", whose bugs were found in the OSS-Fuzz issue tracker. These results were unexpected given that these programs have been already tested extensively during the tool competition hosted by this conference more than a year ago, especially the "ASSIMP" projects that is responsible for the majority of them.

These results show the effectiveness of continuously testing and maintaining up to date your projects, especially considering that FuzzBench tests these projects on older versions than the one used for these tests. \ziosaba{what else?}






\newpage
\subsection{Case study: ASSIMP}
The \textit{"Open Asset Import Library"} (abbreviated \textit{ASSIMP}) \cite{assimp} is a cross-platform library to import 3D models into a shared, in-memory immediate format, also providing a common API for different 3D asset file formats.

Its analysis yielded a total of 160 bugs, 96 of which were unreported, divided as follows: 1 stack-buffer-overflow, 1 stack-overflow, 71 heap-buffer-overflows, 20 crashes and 3 undefined behaviors. The stack-buffer-overflow error was related to a \verb|memcpy| invocation not properly checking the length of the involved buffers, eventually overriding the stack and causing a segmentation fault. The stack-overflow was triggered by an \verb|open| function triggered in a never-ending loop, eventually leading to stack exhaustion. The heap-buffer-overflows and crashes were mostly related to improper sanitizations of user inputs when performing memory accesses, eventually also leading to segmentation faults. All the aforementioned scenarios are some of the most common methodologies that malicious people use when exploiting a program in order to gain control of your system, therefore fixing them would be crucial.

\matteo{The text below can be used if they don't answer by the time you have to submit your thesis (a plausible scenario). If they do, then we can change it, but at least you have a backup plan. Your old text is still commented down here}

We reported all the bugs we found with a (private) security report through the Github platform this project uses. At the time of writing, the developers have not yet publicly acknowledged the existence of those bugs, but they have started providing fixes to them. Given the high number of reported bugs, it is reasonable to expect the bug fixing process to take some time, but it will eventually lead to a much safer and, hopefully, bug-free library. 


% \ziosaba{Waiting for response, add more later...}


% Given the importance of this project, as its popularity and widespread usage allowed it to enter the strict OSS-Fuzz eligibility requirements, we argue it is unacceptable to have this many unresolved bugs. \matteo{There is no need to be this harsh on them, it can be seen as unprofessional at the very least}







\newpage
\section{Developers' responses}
This study has shown that developers have different approaches to bug reporting. 

The OSS-Fuzz reports were submitted in September, and at the time of writing there are still just under 1/3 of the bugs discovered that have yet to be acknowledged by their respective developers. 
As previously shown, most of them were related to use-of-uninitialized-memory and undefined behaviors, but the reports received mixed responses from the developers. The majority either answered directly after the problem was fixed, or acknowledged the correctness of the bugs and the content of the report while promising to fix them in future versions, a sub-optimal approach that shows how many open-source developers either do not have enough time and resources to fix them, or they simply do not care about reports on trivial bugs that are not security relevant and postpone them (sometimes indefinitely). 
There were some cases where the reports were accepted but with negative interactions, as the developers acknowledged the bugs but reprimanded the submitter: this was either because the work presented was effectively a duplicate of OSS-Fuzz and therefore useless (which indeed it was not), or because they thought the person who wrote them wanted to "collect yet another bug report" at their expense. Unfortunately, this highlights a bad practice when it comes to bug reporting: some developers expect the person who reports a bug to also provide a complete analysis and/or fix to lessen their work, which would obviously require extensive skills and knowledge of the project codebase, while some users send general or incomplete bug reports simply for the sake of adding a new bug under their name, which many developers consider to be harmful to their popularity because a buggy program will eventually become less trusted.
Finally, there were also few cases where the submitted reports were rejected or simply ignored altogether. For example, when presented with a series of UUM bugs found by Valgrind, one developer replied by stating that configuring the tool to suppress a specific set of (apparently) false-positive errors from a specific list removed all the errors reported. In another case, the report was rejected because the developer did not trust the origin of the resources provided (i.e. buggy testcase and binary tested) for his own security. There was also an instance where the developer rejected the report because the provided testcase was too much specifically crafted to be a corner-case, in his opinion not effectively a real scenario that could pose a threat and therefore will not be fixed.  

The FuzzBench reports were submitted during December, and at the time of writing only the developers of "ASSIMP" have yet to acknowledge the bugs reported. As previously mentioned, this project produced a non-trivial amount of bugs, promptly notified in a secure manner and suggesting a "High" priority for the related bug report. To help the developers, the report also attempted to categorize and group all bugs discovered: although reporting this many bugs at once would obviously put a heavy burden on the developers, we also observed that this particular report was postponed in favor of smaller bug reports that were sometimes overlapping with the bugs discovered by this work, and we argue the correctness of this behavior. \ziosaba{rewrite maybe, not sure about that}    


\paragraph{Perceived priority} How developers perceive a bug report's priority varies greatly, and this difference becomes even more apparent when considering different kinds of bugs. In general, most bugs reported using ASan have been confirmed (at most) within 2 week of being reported, and then triaged and fixed shortly after. Using memory addresses incorrectly in a program will most likely cause problems on the stack and heap, easily leading into heap-based and stack-based corruption, and both scenarios are known to be some of the most popular attack vectors exploited by malicious people. Similarly, also most memory-related bugs reported using Valgrind have been confirmed and fixed in a matter of weeks. However, almost all bugs reported using UBSan have been ignored, with few exceptions: this is because undefined-behavior bugs often originate from basic arithmetic and logic operations, with a common belief that such mistakes are unlikely to be too harmful and can be addressed later. A simple search on the CVE database \cite{cve} using keywords like "integer overflow", "privilege escalation" and "remote command execution" produced thousands of results, showing how even the most basic, trivial and simple mistakes could have catastrophic consequences, therefore the behavior exhibited by the developers on this kind of bugs is unacceptable. 

Overall, most reports produced were accepted and fixed (albeit not always cordially), but this work highlighted how open-source developers tend to focus on bugs that could pose a real threat to their product, rather than polishing and fixing small bugs that could still lead to unexpected behavior or unwanted interactions, ultimately leading to a worse user experience. Moreover, as discussed earlier, there is a margin of distrust between developers and users, which should be one of the key factors in the development of open-source projects (recall \ref{opensource}), but instead has gained a negative reputation due to bad habits and perhaps too many expectations from both sides.












\newpage
\section{Discussion}
This section contains a discussion of the results from the autonomous fuzzing infrastructures evaluated in this thesis, i.e. OSS-Fuzz and FuzzBench.

\paragraph{OSS-Fuzz.} Albeit obtained from a small subset of projects that are not representative of the entire campaign, the results obtained show that this framework is missing a relevant number of bugs. Furthermore, although not shown in Table \ref{ossfuzz-table}, less than half of the projects tested included AFL++ as one of the fuzzing engines used for the tests: it is interesting to note that all the projects that used AFL++ yielded (on average) less bugs than the others, which again highlights why it is considered the current state-of-the-art fuzzer.

The OSS-Fuzz results were indeed unexpected as one would assume that relying on this infrastructure, especially given the organization running it and the scale of such campaign, would be a fool-proof strategy to ensure that your program is tested correctly and efficiently. The reasons why this work discovered so many bugs are up for debate, as the most inner workings of OSS-Fuzz and its back-end ClusterFuzz are not publicly available in their entirety, probably for security reasons; however, it is apparent that design and implementation choices as well as open-source developers involvement are all factors that contributed to the following key hypothesis.

A trivial explanation lies in the developers' choices when it comes to selecting which sanitizers and fuzzing engines will be used for the tests. Regarding the sanitizers, one would expect to find many bugs of a certain type when introducing a sanitizer that is not already being used by OSS-Fuzz. Also fuzzing engines, similarly to sanitizers, use different approaches and strategies when testing a program: given that each fuzzer generates different new testcases and produces different coverage, using the same corpus with a different combination of fuzzer and sanitizers may produce substantially different results. 

Another reason may lie in the algorithms used by ClusterFuzz to manage the fuzzing queues. Each time a project undergoes a fuzzing session, it will produce a set containing all the interesting input discovered (i.e. produced new coverage) along with any new testcases generated by the fuzzer, that is then manipulated and processed to form what will be the the corpus used as input for future fuzzing sessions. Among the many operations performed, \textit{bug deduplication} and \textit{pruning} are essential to make sure that the size of this corpus does not explode over time. We already established how ClusterFuzz performs \textit{deduplication} (i.e. "error type" and the last 3 stack entries), and also mentioned that there is no general solution to this process that does not involve some degree of information loss.
The same can be said about \textit{pruning}, which is the process of removing all unnecessary and non-relevant inputs from a corpus. ClusterFuzz employs several \textit{"Multi-Armed Bandit (MAB)"} \cite{mab} algorithms, whose statistical explanation will be omitted for the sake of simplicity, in the following way: at the end of each fuzzing session, ClusterFuzz must decide whether to prioritize inputs that caused bugs or inputs that produced new coverage, using regress knowledge and implementation choices as reference. Given that ClusterFuzz's developers prioritize code coverage, the resulting fuzzing queue will contain mostly that type of inputs, implying that when pruning is performed it will also most likely remove some inputs causing bugs from future fuzzing queues.

Related to the problem of pruning, there is also the versioning problem.
Let's assume that it exists a specific input on the current version of the program that causes a bug. Then, the developers release a new version introducing some changes, including a fix for said bug, and the input is automatically tested (successfully) and removed by ClusterFuzz as it is not relevant anymore. Patching a single input does not necessarily mean that particular type of error has been permanently is fixed: given that the input has been removed from the queue, unless another input on future fuzzing queues reproduces the problem, it persists. This is the reason why keeping old inputs that caused bugs and testing old corpora on newer versions is useful, as it is not uncommon for developers to reintroduce old bugs during changes, and they might produce interesting and unexpected results.

Finally, we mention how ClusterFuzz handles bugs that cannot be reliably reproduced \cite{unreliable}: \textit{"ClusterFuzz does not consider testcases that do not reliably reproduce as important. However, if a crash state is seen very frequently despite not having a single reliable testcase for it, ClusterFuzz will file a bug for it. When ClusterFuzz finds a reliably reproducible testcase for the same crash state, it creates a new report and deletes the older report with the unreliable testcase."} During this analysis, there were few inputs that did not produce a bug consistently (i.e. the bug did not appear on all execution with the same testcase as input), something that was obviously mentioned in the reports containing them. Given that it is unknown how many times a crash must be "seen" by ClusterFuzz before a generic bug report for it is produced, it is possible that the reports generated during this work may be related to bugs already known by ClusterFuzz but that were not appearing enough times to be considered relevant.

Overall, the results shown in Table \ref{ossfuzz-table} proved the effectiveness of combining the use of multiple sanitizers (or at least as many as possible) with different fuzzing engines so that there are no missed bugs during the continuous fuzzing process proposed by OSS-Fuzz. In addition, the maintainer of this framework should periodically review how fuzzing queues are managed and tested, to ensure that there are no issues in their testing pipelines and that no potential bugs are kept publicly available by users downloading such queues. Finally, the project developers should also periodically check the correctness of the settings used for the tests, to ensure that all tests are always performed to be as effective and accurate as possible.  

\paragraph{FuzzBench.} The standard FuzzBench benchmarks and the additional benchmarks from the "SBFT '23" conference were tested using their OSS-Fuzz counterparts, as we have already mentioned that FuzzBench performs tests on predefined checkout versions and using the same set of input corpora across different executions for reference purposes. Even when tested with OSS-Fuzz on latest version, all default benchmarks did not produce any new bugs, which is certainly an unexpected but welcome result as it shows the effectiveness of continuous and rigorous testing. In fact, although not shown in Table \ref{fuzzbench-table}, almost all the projects tested used all the available fuzzing engines provided by ClusterFuzz (i.e. libFuzzer, AFL++ and Honggfuzz), which shows how testing a program with a variety of fuzzers and \textit{ensemble fuzzing} provides overall better results due to the heterogeneity of the testcases used. However, all bugs discovered thanks to FuzzBench came from the set of projects used in the "SBFT '23" conference, which came as a surprise as one would think that being them the object of a tool competition also implied that they were thoroughly tested, revised and fixed after the competition ended. This is especially true for the "ASSIMP" project, as we argue it is unacceptable for a project that has managed to satisfy the rigid OSS-Fuzz requirements of being a highly trusted and regarded project by the entire open-source community to also produce so many bugs and crashes.

In general, when testing older versions of a program, it is good practice to check whether there is an older bug that is still present in the latest version, as it is not uncommon for developers to make changes to the code that mistakenly reintroduce a previously fixed bug. This statement should apply to any automated fuzzing framework, such as those evaluated in this thesis, as well as events like the fuzzing-oriented "SBFT '23", where those responsible for the fuzzing tool challenge failed to do so and left many valuable bugs open to anyone but unreported. 

\chapter{Conclusions and Future Work} \label{chap_5}
\ \\

In this last chapter, we will summarize the work presented in this thesis. After that, we will present some directions for future work that could be explored to further improve our solution and the impact it can have in malware analysis research.

\section{Conclusion}
This work first introduced the concepts of fuzzing and sanitizers, a popular bug-detection approach in testing environments that has proven to be particularly effective, especially when combined with the "Continuous-Fuzzing/Continuous-Integration" pipeline that many modern organizations employ when developing their products. Then, we introduced autonomous fuzzing frameworks, how they work and described the chosen frameworks analyzed. Followed a description of the methodology used to analyze said frameworks along with how bugs were collected, analyzed and reported. Concludes a thorough analysis of the results, discussing how developers perceived the bug reports created and the implications of the behaviors observed with some case studies.

This study highlighted the popularity of autonomous fuzzing infrastructures and their effectiveness, but also that the lack of standardized approaches and design trade-offs are major contributors to their accuracy when it comes to automatically detecting and reporting bugs. Although the overall number of projects analyzed is not representative of the entire campaigns, the applied methodology still managed to discover several hundreds of previously unreported bugs, showing that the workflow adopted by the analyzed campaigns presents flaws that lead to overlooked bugs and potential vulnerabilities.

In today's world, where software development techniques are constantly changing and improving, it is crucial to ensure that all aspects of software development progress at the same pace, including the "Testing" phase: many organizations rely on automated testing, a time-efficient and effective solution, but that comes with its shortcomings as highlighted by this work. Therefore, while relying on autonomous testing techniques and infrastructures has proven its efficacy in time, developers should monitor, configure and employ these tools regularly and keep them up to date to ensure that also the testing environments can be as accurate and productive as possible.


\newpage
\section{Future Work}
The results shown by this work highlighted how using different combinations of fuzzing engines and sanitizers produced new and previously unseen bugs, which if properly fixed ultimately lead to a more refined and stable product, meeting the objective of many modern organizations. However, it is also important to notice that most autonomous fuzzing frameworks have limitations due to design choices and resources trade-offs, which have been considered acceptable after rigorous accuracy and effectiveness analyses. Therefore, applying a single solution to all of them would be suboptimal. Nonetheless, keeping the most inner workings not easily accessible or private will obviously limit the developers' understanding of such infrastructures, making them unable to provide insightful help or suggestions towards their improvements. 

\matteo{Expand this work to other fuzzing frameworks? We found some issues that are general (e.g. use as many sanitizers as possible), but also issues that are specific to the frameworks we tested: other frameworks might have different issues specific to them}
\ziosaba{Tutti i framework analizzati hanno dei limiti, però è anche vero che (come ho provato ad abozzare sopra) c'è un problema relativo al fatto che se faccio un certo tipo di fuzzing non posso farne un altro, o anche che ci sono del limiti relativi all'utilizzo di risorse che a casisitche in cui si possono perdere dei bug. Mi verrebbe in mente di scrivere una cosa del tipo "questi framework dovrebbero fornire delle linee guida generali per consentire agli sviluppatori di massimizzare i tool e le funzionalità fornite", o qualcosa del genere}



Overall, fuzzing is a technique with no general guidelines: each developer tests their programs as they see fit, using general common sense but also following potentially wrong beliefs, and this oftentimes lead to missing bugs in the process. 

\matteo{Provide general guidelines for fuzzing frameworks. Right now there is none, users just adhere to common sense; we could study and propose guidelines that prevent the issues we witnessed.}
\ziosaba{Anche qui: fare fuzzing richiede conoscenze, skill e (potenzialmente) ricompilare cose che non si possono ricompilare (MSan con le librerie) e quindi c'è una limitazione relativa alla versatilità dei tool, ma non so se ha senso come "lavoro futuro". Oltretutto, tornando al discorso di prima, ci sarebbe da fare delle guideline del tipo "testa tutto con tutto",ma se non hai le conoscenze e le skill per compilare i progetti in tanti modi diversi e secondo le richieste di un tool specifico dovresti creare delle guide speciifche che ti spiegano tutto per filo e per segno, bho non mi convince molto.}

\matteo{can we think of something else? maybe something related to sanitizers}



\backmatter
\phantomsection
\bibliographystyle{sapthesis} % BibTeX style
\bibliography{bibliography}




\end{document}